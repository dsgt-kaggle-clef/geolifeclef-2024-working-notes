\section{Conclusions}

In this study, we addressed the multi-label classification challenge of GeoLifeCLEF 2024, which aims to predict the presence or absence of plant species at specific locations based on spatial and temporal remote sensing data. 
We explored using a compressed version of the remote sensing data to train deep learning models, with varying levels of success.
We take advantage of the geospatial nature of the data by building a neighborhood model with locality-sensitive hashing.
Predictions from the neighborhood model perform better than some of the simplest frequency models made by the competition organizers.
The neighborhood model is used as part of a self-supervised embedding model that learns a low-dimensional representation of the data that is effective for classification.
Despite poor performance on the leaderboard, some of the ideas presented in this working note have potential for future work and have not been fully explored.
Source code and models are available at \url{https://github.com/dsgt-kaggle-clef/geolifeclef-2024}.

\section*{Acknowledgements}

Thank you to Professor Patricio Vela for supervising the project for Anthony Miyaguchi's ECE8903 Special Problems course at Georgia Tech.
Thank you to the DS@GT CLEF group for access to cloud computing resources through Google Cloud Platform, and for a supportive environment for collaboration.