\copyrightyear{2024}
\copyrightclause{Copyright for this paper by its authors.
  Use permitted under Creative Commons License Attribution 4.0
  International (CC BY 4.0).}

\conference{CLEF 2024: Conference and Labs of the Evaluation Forum, September 9-12, 2024, Grenoble, France}

\title{Tile Compression and Embeddings for Multi-Label Classification in GeoLifeCLEF 2024}

\author[1]{Anthony Miyaguchi}[
orcid=0000-0002-9165-8718,
email=acmiyaguchi@gatech.edu,
]
\cormark[1]
\author[1]{Patcharapong Aphiwetsa}[
email=paphiwetsa3@gatech.edu,
]
\author[1]{Mark McDuffie}[
email=mmcduffie8@gatech.edu,
]

\address[1]{Georgia Institute of Technology, North Ave NW, Atlanta, GA 30332}
\cortext[1]{Corresponding author.}

\begin{abstract}
We explore methods to solve the multi-label classification task posed by the GeoLifeCLEF 2024 competition with the DS@GT team, which aims to predict the presence and absence of plant species at specific locations using spatial and temporal remote sensing data. 
Our approach uses frequency-domain coefficients via the Discrete Cosine Transform (DCT) to compress and pre-compute the raw input data for convolutional neural networks.
We also investigate nearest neighborhood models via locality-sensitive hashing (LSH) for prediction and to aid in the self-supervised contrastive learning of embeddings through tile2vec.
Our best competition model utilized geolocation features with a leaderboard score of 0.152 and a best post-competition score of 0.161.
Source code and models are available at \url{https://github.com/dsgt-kaggle-clef/geolifeclef-2024}.
\end{abstract}

\begin{keywords}
  GeoLifeCLEF,
  LifeCLEF,
  remote sensing,
  contrastive learning,
  multi-label classification,
  tile2vec,
  discrete cosine transform,
  locality-sensitive hashing
\end{keywords}
