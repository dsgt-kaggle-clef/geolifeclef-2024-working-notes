\section{Introduction}

GeoLifeCLEF \cite{geolifeclef2024} is a task organized within the LifeCLEF lab \cite{lifeclef2024} at the CLEF 2024 conference, with the objective of predicting which plant species are present and absent in specific locations given spatial and temporal remote sensing data. 
Modeling species density distributions can be helpful in biodiversity management and conservation.

We explore methods to solve the posed multi-label classification task and incorporate unsupervised methods to build useful representations from the data. 
We propose a pipeline that pre-computes tiles from raw GeoTIFF images and stores a compressed version on disk to speed up the training process.
We utilize metadata to build baseline geolocation models and indices for nearest-neighbor queries.
Our models utilize convolutional neural networks to exploit spatial information, spectral representations, and co-located bands of remote sensing data.
We also explore the use of unsupervised methods to learn representations for knowledge transfer between two different datasets with similar semantics.
